\newpage
\subsection{\underline{Random Selection}}
\begin{itemize}
\item Each individual has a corresponding genome
\bi
\item This genome is represented as a string $s\in A$, where $A$ are the possible characters in a genome
\bi
\item In a real genome, the letters $A, C, G$, and $T$ (or $U$) are used to represent the real nucleotide bases Adenine, Cytosine, Guanine, and Thymine (or Uracil, present in RNA). So in this case, $A=\{A, C, G, T, U\}$
\item However, our implementation of a genome will be using binary digits to encode genetic information. So in our model, $A=\{0,1\}$
\ei
\item So we can let the set of all individual fitnesses be $F_g=\{f_1, f_2, \ldots, f_p\}$, where $g$ is a given generation.
\item As we are looking at a population of $500$ creatures, we have $p=500$. This number will be held constant across all generations.
\item However, this fixed population value can be picked to be any other value. So let us initialize a variable $p$ to represent the total number of individuals in the population of creatures. 
\ei
\item Offspring:
\bi
\item An individual produces offspring with a probability of $\frac{f_i}{n\bar{f}}$
\item I.e. An individual is only allowed to reproduce if its fitness value is greater than the mean fitness value
\item After, individuals that are selected to reproduce are ordered by descending individual fitness.
\item Starting from the top of the list, pairs of two are 
\bi
\item This process mimics sexual selection
\ei
\item If there are an odd number of individuals chosen to reproduce, a random individual will be removed from the mating pool, simulating natural conditions
\item Number of offspring is equal
\ei
\item Survival:
\bi
\item At the end of a generation, all individuals from one generation are killed off - only offspring survive into the next generation.
\ei
\end{itemize}
