\newpage
\subsection{\underline{Fitness Function}}

In our simulation, we want to provide incentive for individuals who stay alive the longest. This models reality, as individuals who stay alive longer are able to reproduce more, thus have a greater biological fitness.

\ms
\nid At the end of one generation, we have a measure of performance from each individual. So let the fitness function be defined as the following, where $t$ is the time in seconds an individuals stays alive:

$$f_i(t)=t^2$$

\ms
\nid This function was chosen to disproportionately incentivizes genetics from individuals which had stayed alive longer. This means that over many generations, we should see a trend towards genetics from individuals who stayed alive the longest, i.e. genetics indicating high biological fitness. 

\ms
\nid After running this calculation for all $p$ individuals in a generation, we will have a set of all individual fitnesses:

$$
F_g=\{f_1, f_2, f_3, \ldots, f_{p}\}
$$

\nid To find the proportion of an individual's contribution to the gene pool of a population, we find the ratio of an individual's fitness divided by sum of all the individual fitness values of the population. So if $W_i$ is the weight of fitness each individual holds in the population's gene pool, then:

$$
W_i=\frac{f_i}{\mathlarger{\sum_{k=1}^{p} f_k}}
$$

\nid From here, we can determine how many offspring an individual will produce: 
$\frac{f_i}{\bar{f}}$
\bi
\item A ratio we can use in later calculations, proportional to the fitness of an individual in relation to the mean individual fitness in the population.
\ei

\ms
\nid Additionally, we will create a function which takes in a fitness ranking of an individual, and outputs the number of offspring that an individual can have.