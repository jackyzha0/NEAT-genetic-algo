\newpage
\subsection{\underline{Overview}}

\begin{itemize}
\item Initial setup, and random selection:
\begin{itemize}
\item Population: a set of individuals, which we can determine the size of
\item Individual: has a unique genome, which is represented as a string with elements from a genetic alphabet $A=\{0,1\}$
\item Each number or section of numbers encodes a particular trait
\end{itemize}
\item Fitness function:
\begin{itemize}
\item Fitness values could be replaced by ranks.
\item E.g. if there are 32 animals in a population, each animal gets a ranking from numbers 1 to 32 based on what we deem as the most fit characteristics
\end{itemize}
\item Crossover:
\begin{itemize}
    \item Choose a pair of individuals who will reproduce
    \item Choose a position from 1 to $L - 1$ randomly
    \item Cut both genomes at this point and reattach the parts of each parent
    \item Move new pairs to population
\end{itemize}
\item Mutation:
\bi
    \item Choose a small probability
    \item For all individuals and each position in their respective genes,
    1 to $L$, roll the small probability
    \item If probability met, flip the gene
\ei
\item Termination condition:
\bi 
    \item After repeating the above process we need to know when to stop
    \item Find optimal solution (fitness sufficiently close to optimal)
    \item Fixed number of generations reached or time/memory limit exceeded
    (safety conditions)
    \item Diversity of the population has vanished
    \item Plateau of the highest ranking solution
\ei
\end{itemize}