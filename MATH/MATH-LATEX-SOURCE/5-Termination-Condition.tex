\newpage
\subsection{\underline{Termination Condition}}

\bi
\item The most commonly used termination conditions used within genetic
algorithms are as follows:
\bi
\item After repeating the above process we need to know when to stop
\item Find optimal solution (fitness sufficiently close to optimal)
\item Fixed number of generations reached or time/memory limit exceeded
(safety conditions)
\item Diversity of the population has vanished
\item Plateau of the highest ranking solution
\ei
\item For our implementation, we decided to use two termination conditions.
The first is if all the individuals die in which we must obviously move
on to the next generation. Second, we will start the next generation if the 
individuals survive for a certain amount of time.
\item There are vast and fairly obvious differences in our simulation as
compared to nature here. However this is a result of genetic algorithms 
often having a purpose like to solve a problem, while in nature individuals
try to optimize their adaptation to the environment.
\ei

\bi
\item Note: At the end of a generation, all individuals from one generation are killed off - only offspring survive into the next generation.
\ei